\documentclass[draft]{agujournal2019}
\usepackage{url} %this package should fix any errors with URLs in refs.
\usepackage{lineno}
\usepackage[inline]{trackchanges} %for better track changes. finalnew option will compile document with changes incorporated.



%% References and call to Sections
%\usepackage[colorlinks=true]{hyperref}
%\usepackage[nameinlink,capitalise]{cleveref}
%\crefname{equation}{}{}


\linenumbers
%%%%%%%
% As of 2018 we recommend use of the TrackChanges package to mark revisions.
% The trackchanges package adds five new LaTeX commands:
%
%  \note[editor]{The note}
%  \annote[editor]{Text to annotate}{The note}
%  \add[editor]{Text to add}
%  \remove[editor]{Text to remove}
%  \change[editor]{Text to remove}{Text to add}
%
% complete documentation is here: http://trackchanges.sourceforge.net/
%%%%%%%
\addeditor{TP}

\draftfalse


%%User-defined packages

\usepackage{amssymb}
\usepackage{amsmath}
\usepackage{graphicx}
\graphicspath{{figures/}}
\usepackage{siunitx}
\usepackage[T1]{fontenc}

\usepackage{soul}
\usepackage[abs]{overpic}
\usepackage{bm}
\usepackage{algorithm,algorithmic}


%%COMMANDS
\newcommand{\pard}[2]{\frac{\partial #1}{\partial #2}}
\newcommand{\vct}[1]{\underline{#1}}
\newcommand{\tsr}[1]{\underline{\underline{#1}}}
\newcommand{\R}{{\mathbb R}}
\newcommand{\calT}{{\mathcal T}}
\newcommand{\Tcav}{{\mathcal T}_{\text{\rm cav}}}
\newcommand{\Gammab}{\Gamma_{\text{b}}}
\newcommand{\Gammau}{\Gamma_{\text{u}}}
\newcommand{\dv}{\text{\rm div}}
\newcommand{\Id}{\text{\rm Id}}
\newcommand{\I}{\text{\rm I}}
\newcommand{\tr}{\text{\rm tr}}
\renewcommand{\d}{\text{\rm d}}
\renewcommand{\o}{\text{\rm o}}
\newcommand{\uobs}{\textbf{u}_{\text{\rm obs}}}
\newcommand{\JLS}{J_{\text{\rm LS}}}
\renewcommand{\u}{\textbf{u}}
\renewcommand{\v}{\textbf{v}}
\newcommand{\p}{\textbf{p}}
\newcommand{\n}{\textbf{n}}
\newcommand{\x}{\textbf{x}}
\newcommand{\f}{\textbf{f}}
\newcommand{\bz}{\textbf{0}}
\newcommand{\btheta}{{\bm\theta}}


\journalname{Geophysical Research Letters}


%Insructions

% Research letters are articles on major advances in all major geoscience disciplines.  Research letters are only available in Geophysical Research Letters, Geochemistry, Geophysics, Geosystems, and Earth and Space Science.  Letters should have broad and immediate implications in their discipline or across the geosciences. Research letters are limited to 12 publication units. Submitted manuscripts longer than 12 publication units will be returned for shortening

% The formula for publication units (PU) = number of words/500 + number of figures + number of tables. Word count includes abstract, text, in-text citations, figure captions, and appendices. Word count excludes title, author list and affiliations, plain-language summary, text inside the table, Open Research section, references, and supporting information.


%%TODO : list of remaining taks

%%Supporting informations
%	add explanations on derivatiion of J'
%	include brief description to overview of how magmaOpt operates(reference to sotuto)
%
%% github
%	creat clan repo to put magma opt
%	include installation instruction (FreeFem install, advect install, conda environnement install)
%	Modify code to let it run on every computer
%	include an optimization example
%	
%%Main text
%	
%	
%	
%% Tests
%
%	Do tests on behavior of algo


%%%%%%%WORD COUNT :  3600 w = 7 publication units
%%%%%%%FIGURE COUNT : 4 = 4 PU
%%%%%%% TOTAL PU : 11

\begin{document}

%%%%%%%%%%%%%%%%%%%%%%%%%%%%%%%%%%%%%%%%%%%%%%%
%  TITLE
%
% (A title should be specific, informative, and brief. Use
% abbreviations only if they are defined in the abstract. Titles that
% start with general keywords then specific terms are optimized in
% searches)
%
%%%%%%%%%%%%%%%%%%%%%%%%%%%%%%%%%%%%%%%%%%%%%%%

% Example: \title{This is a test title}

\title{A shape optimization approach towards improving the understanding of magmatic plumbing system}

%%%%%%%%%%%%%%%%%%%%%%%%%%%%%%%%%%%%%%%%%%%%%%%
%
%  AUTHORS AND AFFILIATIONS
%
%%%%%%%%%%%%%%%%%%%%%%%%%%%%%%%%%%%%%%%%%%%%%%%

% Authors are individuals who have significantly contributed to the
% research and preparation of the article. Group authors are allowed, if
% each author in the group is separately identified in an appendix.)

% List authors by first name or initial followed by last name and
% separated by commas. Use \affil{} to number affiliations, and
% \thanks{} for author notes.
% Additional author notes should be indicated with \thanks{} (for
% example, for current addresses).

% Example: \authors{A. B. Author\affil{1}\thanks{Current address, Antartica}, B. C. Author\affil{2,3}, and D. E.
% Author\affil{3,4}\thanks{Also funded by Monsanto.}}

\authors{Théo Perrot\affil{1}, Freysteinn Sigmundsson\affil{2}, Charles Dapogny\affil{3}}


\affiliation{1}{Mechanical engineering department, Ecole Normale Supérieure Paris-Saclay}
\affiliation{2}{Institute of Earth Sciences, University of Iceland}
\affiliation{3}{Jean Kuntzmann Institute, CNRS, Grenoble-Alpes University}




% Corresponding author mailing address and e-mail address:

% (include name and email addresses of the corresponding author.  More
% than one corresponding author is allowed in this LaTeX file and for
% publication; but only one corresponding author is allowed in our
% editorial system.)

% Example: \correspondingauthor{First and Last Name}{email@address.edu}

\correspondingauthor{Théo Perrot}{theo.perrot@ens-paris-saclay.fr}



%%%%%%%%%%%%%%%%%%%%%%%%%%%%%%%%%%%%%%%%%%%%%%%
% KEY POINTS
%%%%%%%%%%%%%%%%%%%%%%%%%%%%%%%%%%%%%%%%%%%%%%%
%  List up to three key points (at least one is required)
%  Key Points summarize the main points and conclusions of the article
%  Each must be 140 characters or fewer with no special characters or punctuation and must be complete sentences

% Example:
% \begin{keypoints}
% \item	List up to three key points (at least one is required)
% \item	Key Points summarize the main points and conclusions of the article
% \item	Each must be 140 characters or fewer with no special characters or punctuation and must be complete sentences
% \end{keypoints}

\begin{keypoints}
\item We present a novel approach to allow for assessment of volcanic magma domains shape based on level-set shape optimization.
\item It relies on numerical finite element models iteratively modified to minimize the discrepancy to observed surface displacements
\item We found strong dependence of best solution to initialization when benchmarked with synthetic data but application on data from Svartsengi 2022 inflation outputed relevant results.
\end{keypoints}

%%%%%%%%%%%%%%%%%%%%%%%%%%%%%%%%%%%%%%%%%%%%%%%
%
%  ABSTRACT and PLAIN LANGUAGE SUMMARY
%
% A good Abstract will begin with a short description of the problem
% being addressed, briefly describe the new data or analyses, then
% briefly states the main conclusion(s) and how they are supported and
% uncertainties.

% The Plain Language Summary should be written for a broad audience,
% including journalists and the science-interested public, that will not have 
% a background in your field.
%
% A Plain Language Summary is required in GRL, JGR: Planets, JGR: Biogeosciences,
% JGR: Oceans, G-Cubed, Reviews of Geophysics, and JAMES.
% see http://sharingscience.agu.org/creating-plain-language-summary/)
%
%%%%%%%%%%%%%%%%%%%%%%%%%%%%%%%%%%%%%%%%%%%%%%%

%% \begin{abstract} starts the second page

\begin{abstract}
In volcano geodesy, pressure sources in volcano roots responsible for surface movements are inverted using ground deformation data after defining a forward parametric model for the source. Such models are most of the time relying on predefined shape for the source, which can limit their accuracy. On the contrary, we propose here a shape optimization method to invert for pressure sources without any prior shape assumption. With that flexibility, the optimal shape of a pressurized magma body is determined by minimizing the discrepancy between observed and modelled displacement. We explore the capabilities of this approach with synthetic data first for validation and then apply it to observed ground deformation at the Svartsengi volcanic system in Iceland, demonstrating its potential to improve volcanic hazard assessment after maturation with further work.
\end{abstract}

\section*{Plain Language Summary}


Reservoirs beneath volcanoes contain pressurized magma that can feed eruptions when it breaks through. Because the pressure on the surrounding rocks is so high, the reservoir causes changes in ground motion at the surface that can be detected by instruments. Geophysicists use these measurements to reconstruct the position, orientation, and pressure of reservoirs in the crust and to determine the likelihood of a future eruption. Here we present a new method that could allow them to also reconstruct the shape of these reservoirs, which are often odd and irregular, while current reconstruction methods assume nice and smooth shapes such as spheres. This method is based on shape optimization, a framework widely used in fields such as engineering, where it is used to improve the design of objects or mechanisms. We tested the method on synthetic motion data and on real data from the Svartsengi volcanic system in Iceland. Although our results show that the method needs to be improved before it can be used in operational monitoring, they pave the way for future developments and may help to better reconstruct magma reservoirs.


\note[TP]{https://www.agu.org/Share-and-Advocate/Share/Community/Plain-language-summary}


%%%%%%%%%%%%%%%%%%%%%%%%%%%%%%%%%%%%%%%%%%%%%%%
%
%  BODY TEXT
%
%%%%%%%%%%%%%%%%%%%%%%%%%%%%%%%%%%%%%%%%%%%%%%%


% The main text should start with an introduction. Except for short
% manuscripts (such as comments and replies), the text should be divided
% into sections, each with its own heading.

% Headings should be sentence fragments and do not begin with a
% lowercase letter or number. Examples of good headings are:

% \section{Materials and Methods}
% Here is text on Materials and Methods.
%
% \subsection{A descriptive heading about methods}
% More about Methods.
%
% \section{Data} (Or section title might be a descriptive heading about data)
%
% \section{Results} (Or section title might be a descriptive heading about the
% results)
%
% \section{Conclusions}
    

\section{Introduction}
\note[TP]{Section ok}

In volcano geodesy, inverse problems are central to estimating the position of pressurized magma bodies at depth in volcano roots, using observed crustal deformation as a proxy. 
The displacement is measured by e.g. Global Navigation Satellite System (GNSS) point positioning, leveling campaigns, or interferometry analysis of synthetic aperture radar (InSAR) in a volcanic region \cite{dzurisin2007}. The subsurface processes causing the movement are inferred from these observations. Magmatic sources are modeled as pressurized cavities that deform the surrounding host rocks and cause the surface to move. Various inversion methods based on parametric analytical or numerical models aim at finding the optimal values for the vector of $d$ free parameters $\vct{m} \in \mathbb{R}^d$ of a model. An error function $J(\vct{m})$ is representative of the misfit between the observed displacements and the prediction of the model. $\vct{m}_{opt}$ can then be found using various inversion techniques minimizing $J$: global optimization based on analytic \cite{cervelli2001} or numerical models \cite{hickey2014,charco2014}, Bayesian inference \cite{bagnardi2018,trasatti2022}, or genetic algorithms \cite{velez2011} on analytic models. The choice of the method can be influenced by what is feasible regarding the number of evaluations of $J(\vct{m})$: numerical models handle a complex description of the system, but are computationally expensive compared to analytic models, which on the other hand rely on strong simplifying assumptions \cite{taylor2021}.

However, each of these finite-dimensional optimization methods is limited by the intrinsic assumption of a definite parametric shape for the source. In fact, analytic expressions can be derived for only a few regular shapes such as point source \cite{mogi1958}, finite sphere source \cite{mctigue1987}, or ellipsoidal source \cite{yang1988}, and any numerically generated shape must be parameterized to be inverted. A workaround would be an approach relying on shapes parameterized with more parameters, such as B-splines surfaces, to allow more exploration in the possible shapes. \annote[TP]{This implies optimization within a high-dimension domain, bringing unpleasant phenomena known as the curse of dimensionality}{Ref needed}. The goal of this paper is not to give a definitive answer to these limitations, but rather to lay the first stone for a new approach that overcomes these difficulties.

\subsection{Shape optimization}
\label{sec:th}

Shape optimization generally aims to minimize a cost function depending on the domain. 
This practice is very popular in various disciplines, such as structural mechanics, where one typically wishes to improve the stiffness of a solid structure \cite{bendsoe2004}, 
fluid mechanics, where it is applied to the design of pipes, heat exchangers or flying obstacles \cite{feppon2020}, or again electromagnetism \cite{lucchini2022}. 
Beyond academic investigations, it has aroused a tremendous enthusiasm in industry; nowadays, most Finite Element simulation and design softwares include a shape optimization module \cite{frei2015}, \cite{slavov2019}, \cite{lequilliec2014}. However, the use of these techniques in volcano geodesy is new to the best of our knowledge.

Multiple shape and topology optimization frameworks are available, see e.g. the review in \cite{sigmund2013}. 
One popular strategy describes the design as a density function $\rho$ on a large, fixed computational domain: $\rho$ takes values $0$ and $1$ in the void and material regions, respectively, and intermediate values in between account for a fictitious mixture of both \cite{sigmund2001,bendsoe2004}. One major drawback of this approach is that it does not feature a clear representation of the boundary of the optimized design; in particular, approximations are needed to calculate physical quantities related to the domain at play. To alleviate this issue, we rely on a recent  version of the Level Set method for shape optimization, which benefits from an explicit representation of the boundary at each step of the optimization. 

%%%%%%%%%%%%%%%%%%%%
%%%%%%%%%%%%%%%%%%%%
\section{Method}\label{sec:mag}
%%%%%%%%%%%%%%%%%%%%
%%%%%%%%%%%%%%%%%%%%

This section describes the considered shape optimization problem and its practical implementation. 
For a more complete mathematical background, we refer to e.g. \cite{allaire2021}. 

%%%%%%%%%%%%%%%%%%
\subsection{Presentation of the physical model}
%%%%%%%%%%%%%%%%%%

The region under scrutiny is represented by a fixed bounded domain $D \subset \R^3$, 
which is made of two complementary subdomains $\Omega$ and $\Omega^c :=D \setminus  \overline\Omega$, where:
\begin{itemize}
\item The cavity $\Omega \Subset D$ stands for the magma chamber, whose shape is to be reconstructed. 
Its boundary $\Gamma = \partial\Omega$ is subjected to the force $\f = \Delta P \n$, aligned with the unit normal vector  $\n : \Gamma \to \R^3$ to $\Gamma$ pointing outward $\Omega$, and whose magnitude equals the (given) pressure difference $\Delta P$ between the cavity and the surrounding crust, see Section \ref{sec.Discussion} for a discussion about this point. 
\item The complement $\Omega^c$ of $\Omega$ represents the surrounding Earth crust. It is  filled by a homogeneous, isotropic elastic material. 
The displacement of the bottom side $\Gamma_b$ of $\partial D$ is set to $\bz$ and the other boundary regions of $D$ are free of stress.
\end{itemize} 
This setting is illustrated on Fig. \ref{fig.optmodel} (a).
The displacement $\u_\Omega: \Omega^c \to \R^3$ of the crust in these circumstances is the solution to the system of linearized elasticity:
%If \Delta P >0, the cavity pushes the boundary \Gamma of the cavity to expand
\begin{equation}\label{eq.elas}
\left\{
\begin{array}{cl}
-\dv(Ae(\u_\Omega)) = \bz & \text{in } \Omega^c,\\
\u_\Omega = \bz & \text{on } \Gammab, \\
Ae(\u_\Omega)(-\n) = \f& \text{on } \Gamma, \\
Ae(\u_\Omega)\n = \bz& \text{on } \partial D\setminus \overline{\Gammab}, \\
\end{array}
\right.
\end{equation}
where $e(\u) := \frac12(\nabla \u + \nabla \u^T)$ is the strain tensor induced by a displacement $\u$ and $A$ is the Hooke's law of the crust material. 

\begin{figure}[!ht]
\centering
\begin{tabular}{cc}
\begin{minipage}{0.55\textwidth}
\begin{overpic}[width=1.0\textwidth]{figures/physmodel}
\put(0,3){\fcolorbox{black}{white}{$a$}}
\end{overpic}
\end{minipage}
 & 
 \begin{minipage}{0.4\textwidth}
\begin{overpic}[width=1.0\textwidth]{figures/hadamard}
\put(0,3){\fcolorbox{black}{white}{$b$}}
\end{overpic}
\end{minipage}
\end{tabular}
\caption{\it (a) Sketch of the physical model; (b) Variation $\Omega_{\btheta}$ of $\Omega$.}
\label{fig.optmodel}
\end{figure}

%%%%%%%%%%%%%%%%%%
\subsection{Shape optimization for the reconstruction of the magma chamber}
%%%%%%%%%%%%%%%%%%

In the applications of this article, the shape $\Omega$ of the magma chamber is unknown.
From the datum of observed values $\uobs : \Gammau \to \R^3$ of the displacement of the crust on the upper surface $\Gammau$ of $D$, we intend to identify $\Omega$ as the solution to the following shape optimization problem:
\begin{equation}\label{eq.sopb}
	\min_{\Omega\subset D} \JLS(\Omega), \text{ where } \JLS(\Omega) := \int_{\Gammau} \lvert \u_\Omega- \uobs \lvert ^2 \:\d s,
\end{equation}
featuring the least-square discrepancy between the prediction $\u_\Omega$ of the physical model (\ref{eq.elas}), and the observed displacement $\uobs$ on $\Gamma_u$.
%%%%%%%%%%%%%%%%%%%%%%%%
\subsection{Shape derivatives}
%%%%%%%%%%%%%%%%%%%%%%%%

The treatment of (\ref{eq.sopb}) calls for a notion of derivative for a function $J(\Omega)$ of the domain $\Omega$. In this work, we rely on the boundary variation method of Hadamard, see \cite{allaire2006,allaire2021,henrot2018,murat1976}. 
In short, variations of a reference domain $\Omega$ are considered under the form 
$$\Omega_{\btheta} := (\Id + \btheta)(\Omega), \text{ where } \btheta: \R^3 \to \R^3 \text{ is a ``small'' vector field,}$$
see Fig. \ref{fig.optmodel} (b).
The shape derivative $J^\prime(\Omega)(\btheta)$ of a function $J(\Omega)$ at $\Omega$ is the derivative of the underlying mapping $\btheta \mapsto J(\Omega_{\btheta})$, which produces the following expansion:
\begin{equation}\label{eq.expJtheta}
J(\Omega_{\btheta}) = J(\Omega) + J^\prime(\Omega)(\btheta) + \o(\btheta), \text{ where } \frac{\o(\btheta)}{\lvert\lvert \btheta\lvert\lvert} \xrightarrow{\btheta\to \bz} 0. 
\end{equation}
In practice, $J^\prime(\Omega)(\btheta)$ is used to identify a descent direction $\btheta$, i.e. a vector field such that 
$$J^\prime(\Omega)(\btheta)<0, \text{ so that for a ``small'' step } \tau > 0, \quad J(\Omega_{\tau \btheta}) \approx J(\Omega) + J^\prime(\Omega)(\btheta) < J(\Omega) .$$ 
Intuitively,  the perturbed shape $\Omega_{\tau \btheta}$ performs ``better'' than $\Omega$ with respect to $J(\Omega)$.  

%Overall, this method can be considered a classical iterative gradient descent algorithm. $J$ is first initialized at $J_0$ with an instructed first guess for $\Omega_0$ and then iteratively decreased by moving $\partial \Omega$ of a given step in a given descent direction $\theta: \mathbb{R}^3 \mapsto\mathbb{R}^3 \in W^{1,\inf}$ (the Sobolev space of uniformly bounded functions, \cite{}) chosen using the shape derivative $J'(\Omega)(\theta)$.

The calculation of the shape derivative of the functional $\JLS(\Omega)$ in (\ref{eq.sopb}) is a tedious, but classical issue. It can be realized thanks to the adjoint method, see e.g. \cite{cea1986,plessix2006} and \cite{allaire2004} in this particular mathematical context:  
\begin{equation}\label{eq:jp}
	\JLS^\prime(\Omega)(\btheta) = \int_{\Gamma} v_\Omega \: (\btheta \cdot \n) \: \d s, \text{ where } v_\Omega := - Ae(\u_\Omega):e(\p_\Omega) - \Delta P\:\dv(\p_\Omega),
\end{equation}
and the adjoint state $\p_\Omega$ is the solution to the following problem:
\begin{equation}\label{eq.adj}
\left\{
\begin{array}{cl}
-\dv(Ae(\p_\Omega)) = \bz & \text{in } \Omega^c,\\
\p_\Omega = \bz & \text{on } \Gammab, \\
Ae(\p_\Omega)\n = \bz& \text{on } \Gamma \cup \left( \partial D\setminus (\overline{\Gammau} \cup \overline{\Gammab}) \right), \\
Ae(\p_\Omega)\n = -2 ( \u_\Omega - \uobs )& \text{on } \Gammau. \\
\end{array}
\right.
\end{equation}
An outline of this calculation and a discussion about the nature of the adjoint state $\p_\Omega$ are provided in the supplementary material.

The expression (\ref{eq:jp}) paves the way to a natural descent direction for $\JLS(\Omega)$: 
\begin{equation}\label{eq.descent}
\btheta = -v_\Omega \:\n.
\end{equation}
%
%%%%%%%%%%%%%%%%%%%%
\subsection{Level-set representation}
%%%%%%%%%%%%%%%%%%%%

The magma chamber $\Omega \subset D$ is represented via the Level Set method, see e.g. \cite{osher2006,sethian1999}, and the article \cite{allaire2004} about its introduction in shape and topology optimization. Briefly, $\Omega$ is described as the negative region of a scalar, ``level set'' function $\phi : D \to \R$. 
\begin{equation}\label{eq.LS}
\forall \x \in D, \quad \left\{\begin{array}{cl}
\phi(\x) < 0 & \text{if } \x \in \Omega, \\
\phi(\x) = 0 & \text{if } \x \in \Gamma, \\
\phi(\x) > 0 & \text{if } \x \in \Omega^c.
\end{array}
\right. 
\end{equation}
This idea is illustrated in a two-dimensional situation Fig. \ref{fig.ls} (a).
\begin{figure}[!ht]
\centering
\begin{tabular}{cc}
 \begin{minipage}{0.45\textwidth}
\begin{overpic}[width=1.0\textwidth]{figures/magmals}
\put(0,3){\fcolorbox{black}{white}{$a$}}
\end{overpic}
\end{minipage}
 & 
 \begin{minipage}{0.53\textwidth}
\begin{overpic}[width=1.0\textwidth]{figures/magmamsh}
\put(0,3){\fcolorbox{black}{white}{$b$}}
\end{overpic}
\end{minipage}
\end{tabular}
\caption{\it (a) Graph of a level set function $\phi:D \to \R$ for the cavity $\Omega$; (b) Meshed representation of $\Omega$ (in red), as a submesh of the total mesh of $D$.}
\label{fig.ls}
\end{figure}

The evolution of a domain $\Omega(t)$ through a velocity field $\textbf{V}(t,\x)$, over a time period $(0,T)$ is conveniently captured in terms of an associated level set function  $\phi(t,\cdot)$ (i.e. (\ref{eq.LS}) holds for all $t \in [0,T]$); the latter indeed solves the following advection equation:
\begin{equation}\label{eq.advect}
 \forall t  \in (0,T), \:\x \in D, \quad \frac{\partial\phi}{\partial t}(t,\x) + \textbf{V}(t,\x) \cdot \nabla \phi(t,\x) = 0.
 \end{equation}
In this framework, dramatic changes of $\Omega(t)$ can be accounted for, including topological changes, i.e. merging of holes, or creation of holes.
 
 In our application where $\Omega(t)$ is the sought solution of (\ref{eq.sopb}), 
 the velocity field is the (negative) descent direction $\btheta$ in (\ref{eq.descent}) and the time $T$ stands for the descent step.  

%%%%%%%%%%%%%%%%%%%%%%%
\subsection{Numerical implementation}
%%%%%%%%%%%%%%%%%%%%%%%

Our practical implementation leverages a recent variant of the level set method for shape optimization, introduced in \cite{allaire2014} -- an open source implementation of which is proposed in \cite{dapogny2023}. 
The latter features an additional step at each optimization iteration $n=0,\ldots$, during which remeshing algorithms are used to create a meshed description of $\Omega^n$, as a submesh of the computational domain $D$, see Fig. \ref{fig.ls}

Our numerical workflow is sketched in Alg. \ref{algo.lagevol}. The associated code, named \texttt{magmaOpt}, is freely available online in \cite{perrot2024}, and comprehensive information about its use is provided in the supplementary material.


\begin{algorithm}[ht]
\caption{Shape optimization algorithm for the reconstruction of a magma chamber.}
\label{algo.lagevol}
\begin{algorithmic}[0]
\STATE \textbf{Initialization:} Initial shape $\Omega^0 \subset D$, mesh $\calT^0$ of $D$ a submesh $\Tcav^0$ of which accounts for $\Omega^0$.
\FOR{$n=0,...,$ until stop criterion is met}
\STATE \begin{enumerate}
\item Calculate a level set function $\phi^n : D \to \R$ for $\Omega^n$ on $\calT^n$. 
\item Calculate the state $\u_{\Omega^n}$ and adjoint state $\p_{\Omega^n}$ on $\Tcav^n$. 
\item Calculate a descent direction $\btheta^n$ for $\JLS$, from $\Omega^n$ on $\calT^n$. 
\item Update the level set function by solving the advection equation (\ref{eq.advect}) on $\calT^n$.
\item Create a new mesh $\calT^{n+1}$ of $D$ in which $\Omega^{n+1}$ exists as a submesh.
\end{enumerate}
\ENDFOR
\RETURN Optimized shape $\Omega^n \subset D$ of the cavity.
\end{algorithmic}
\end{algorithm}

The initial geometry is created thanks to the open-source software \texttt{Gmsh} \cite{geuzaine2009}.
At each iteration $n$, the computational domain $D$ is discretized by a mesh $\calT^n$ which encloses a discretization of the actual shape $\Omega^n$ of the magma chamber as a submesh $\Tcav^n$.
The state and adjoint systems (\ref{eq.elas}) and (\ref{eq.adj}) for $\u_{\Omega^n}$ and $\p_{\Omega^n}$ are solved on this mesh by the open-source Finite Element library \texttt{FreeFem} \cite{hecht2012}. A descent direction $\btheta^n$ is then obtained by (\ref{eq.descent}), and a level set function $\phi^n$ for the next iterate $\Omega^{n+1}$ is obtained by solving (\ref{eq.advect}) thanks to the open-source library \texttt{Advect} \cite{bui2012}. The optimization procedure relies on the null-space algorithm \cite{feppon2020a}, which handles efficiently infinite-dimensional problems as well as bounds, equality and inequality constraints.
% \annote[TP]{Because this algorithm allows for slight increase in the cost function, a given number of steps since the last minimum was found is used as termination criterion.}{Check this with Charles}
% \note[TP]{See if more details are needed on the code, its availability in this section}

%%%%%%%%%%%%%%%%%%%%%%
%%%%%%%%%%%%%%%%%%%%%%

\section{Results}

\subsection{Validation on synthetic data}\label{sec:synth}
%%%%%%%%%%%%%%%%%%%%%%
%%%%%%%%%%%%%%%%%%%%%%



\note[TP]{Paragraph below ok}
To validate the proposed method, cross-validation tests were peformed. Synthetic observation data was generated from a known source using the analytical solution provided by \citeA{mctigue1987}, who derived the expression of the surface displacement field caused by a uniformly pressurized spherical cavity embedded in an isotropic and homogeneous elastic medium. \texttt{magmaOpt} was initialized with various arbitrary first guesses for the source shape and position. Other model parameters (elastic constants, internal pressures) were fixed for both synthetic data generation and shape optimization. 




Behaviors to mention

	
\begin{figure}[h!]	
	\centering
	\includegraphics[width=1.1\textwidth]{synthopt.png}
	\caption{ \note[TP]{CHANGE the figure}  . Evolution of error and succesive shapes taken by the magma source during an optimization loop. The initial guess is a flat ellipsoid of semi-axes $r_x=2\si{km},r_y=3\si{km},r_z=1\si{km}$} centred on the true spherical source. The minimum is reached at iteration 82.
	\label{fig:synth}
	
\end{figure}
 
 
\note[TP]{DISCUSS WHAT TO PUT HERE}

 




\subsection{Test case : magma recharge at Svartsengi, 2022}\label{sec:real}

\note[TP]{Intro OK ?}
The shape of a magma chamber was inferred from the ground inflation observed at Svartsengi, South-West of Iceland, during the period from 21 April to 14 June 2022. This event was one of five inflation episodes that preceded dike breaches at the Sundhnúkur crater row, resulting in the partial destruction of Grindavík \cite{sigmundsson2024}.

The observational data utilized consisted of unwrapped InSAR line-of-sight (LOS) displacement maps of the area, obtained from Cosmo SkyMed satellite and available at \citeA{parks2024}. After uniform downsampling and mesh reprojection, both ascending A32 and descending D132 tracks were employed.  
A slight modification of the least-square function was needed because of the peculiar geometry of InSAR data :
\begin{equation}
	 \JLS^{los}(\Omega) :=  \int_{\Gammau} \sum_{i \in {\text{tck}}} 	\alpha_i ( \u_\Omega \cdot \textbf{l}_i - d_i^{\text{\rm obs}} ) ^2 \:\d s
\end{equation}


where $\text{tck}={A125,D132}$ is the list of used tracks. For each track $i$, $\alpha_i$ is its weight (here we choosed $\forall i, \alpha_i= 1$), $\textbf{l}_i$ is the LOS unit vector of the track (giving the look direction of the satellite to project 3D displacment into LOS geometry), and $d_i^{\text{\rm obs}}$ is the observed LOS displacement.


\begin{figure}[h!]
	\centering
	\includegraphics[width=0.8\textwidth]{resinsar.png}
	\caption{a) Convergence plot with embedded zoom. The blue line is the error and the orange line is the evolution of $\tau$. Minima are reached at iteration 128. b,c) Side and top view of the source $\Gamma_{s}$ minimizing $J$. d) Data, model and residuals of the LOS displacements at iteration 128 for the two InSAR tracks A32 (top) and D132 (bottom). Black arrows are heading and looking directions, coordinates are ISN16 \cite{valssson2019} shifted to a local origin $(2529373 E,179745 N)$. }
	\label{fig:rsar}

\end{figure}

The results shown in figure \ref{fig:rsar} are encouraging: after providing an initial guess located at the center of inflation at depth for a sphere of radius RR, the algorithm is able to iteratively change the shape and depth of the magma domain to finally result in a sill-like flattened spheroid whose centroid is located at DD depth. This is consistent with the presumed depth found in the supporting information of \cite{sigmundsson2024}, which performs an analytical model-based inversion. Although the pressure must be fixed, as explained in \ref{sec:th}, the result can be used to compare the final shape of the magmatic intrusion and give a richer insight into it. Here we see interesting features, such as an increasing thickness on the north side, that can't be traced by any other method. The algorithm produces features that we consider to be artifacts, probably due to mesh refinement problems, such as small holes or horn-shaped features.

\note[TP]{Above more detailed characterization of the results is needed as well as comparison with the analytical results}


%%%%%%%%%%%%%%%%%
%%%%%%%%%%%%%%%%%
\section{Discussion}\label{sec.Discussion}
%%%%%%%%%%%%%%%%%
%%%%%%%%%%%%%%%%%

\note[TP]{Discussio complete enough ? Should have more ref ? Might change as well}

This work paves the way for a new class of methods that tackle an unknown geometry of the magmatic domains, thus giving the possibility to explore irregular shapes that are more likely to exist compared to any other usually assumed regular shapes.
However, even if the first results presented are promising, many questions remain to be answered.

First of all, the internal pressure of the chamber $\Delta P$ must be specified prior to the shape optimization, while for conventional parametric inversions, the pressure is a model parameter optimized alongside the position and shape parameters. To adress this limitation, a potential approach would involve two stages in the optimization. A conventional inversion of a parametric model would be run first, giving a pressure and a first educated guess for the position for $\Omega$. Then a more realistic shape could be sought with a shape optimization taking the output of the parametric inversion as an initial guess. Finally, the parametric optimization would be run again using the new shape as input but optimizing only $\Delta P$. This two final steps could be then run succesively until convergence of $\Omega$ and $\Delta P$. While this two-stage procedure is theoretically sound, further investigation and implementation are necessary to determine its practical feasibility.

As showed in part \ref{sec:synth}, the produced design depend strongly on the prior knowledge, namely the intial guess here. Adding constraints is a way to put more knowledge in the inversion, and get more repeatable results. For example, the volume of the source could be constrained to stay within bounds or even to match a certain value. The implemented shape optimization is certainly able to handle constraints as described by \citeA{allaire2021}, and is in practice almost always constrained in other application. The physical meaning of the best shape might benefit from a more constrained problem.

To better understand the influence of data partitioning and variablitiy, additional tests should be run with synthetic data. We can think of tests such as masking part of the surface displacement field, introducing noise and parasitic signals, reducing the number of data points, as is often the case in reality with areas of volcanic systems lacking data coverage (glacier, river, lava, forest) and subjected to perturbations such as atmospheric distorsions.

It is also important to mention that the behaviour of the algorithm is influenced by numerous parameters of varying importance, starting from the discretization length (element size) or the domain extent, to the limits of the step size $\tau$, the regularization length, or the number of iterations allowed by the line search. A systematic study of each of these parameters would be beneficial in assessing the quality of the shape inferred.


\section{Conclusion}

\note[TP]{Conclusion OK ? Check about the role of a conclusion}

We have shown here, to some extent, the relevance of applying shape optimization to identify the likely shapes taken by a magma chamber using as sole information the measured surface displacement field. \annote[TP]{The set of source shapes found with the tests on real was representative of the diversity of behavior to be expected from the algorithm and underlined the intrinsic non-unicity of the solution to the problem, but demonstrated coherence in the results.}{TO modify depending on the results} The test on real data from Svarstengi 2022 inflation showed a concrete case of how the method could be used after being more mature while displaying a credible solution.

This work was intended as an opening to a new method of interest to the field of volcanology rather than a complete solution, and we hilighted the main limitation. Nevertheless, the code \texttt{magmaOpt} provided with this paper can serve as a basis for future modifications and improvements of the method.


The perspectives are many. Besides necessary modifications, such as constraining the problem with more prior knowledge or characterizing the behavior of the algorithm precisely, more sophisticated models could be involved for the inversion. The versatile nature of the finite element method allows the addition of external loads (e.g tectonic stresses, tidal loads, glacier weights), complex topographic geometries and advanced mechanical behavior of the crust (e.g 	plasticity, viscoelasticity, poroelasticity). By addressing these limitations and extending this approach, the accuracy and reliability of volcanic source inference can be improved. Ultimately, future developments may enable better monitoring of volcanic activity, prediction of eruptions, and provide critical support for hazard mitigation strategies.


%Text here ===>>>


%%

%  Numbered lines in equations:
%  To add line numbers to lines in equations,
%  \begin{linenomath*}
%  \begin{equation}
%  \end{equation}
%  \end{linenomath*}



%% Enter Figures and Tables near as possible to where they are first mentioned:
%
% DO NOT USE \psfrag or \subfigure commands.
%
% Figure captions go below the figure.
% Acronyms used in figure captions will be spelled out in the final, published version.

% Table titles go above tables;  other caption information
%  should be placed in last line of the table, using
% \multicolumn2l{$^a$ This is a table note.}
% NOTE that there is no difference between table caption and table heading in the final, published version
%
%----------------
% EXAMPLE FIGURES
%
% \begin{figure}
% \includegraphics{example.png}
% \caption{caption}
% \end{figure}
%
% Giving latex a width will help it to scale the figure properly. A simple trick is to use \textwidth. Try this if large figures run off the side of the page.
% \begin{figure}
% \noindent\includegraphics[width=\textwidth]{anothersample.png}
%\caption{caption}
%\label{pngfiguresample}
%\end{figure}
%
%
% If you get an error about an unknown bounding box, try specifying the width and height of the figure with the natwidth and natheight options. This is common when trying to add a PDF figure without pdflatex.
% \begin{figure}
% \noindent\includegraphics[natwidth=800px,natheight=600px]{samplefigure.pdf}
%\caption{caption}
%\label{pdffiguresample}
%\end{figure}
%
%
% PDFLatex does not seem to be able to process EPS figures. You may want to try the epstopdf package.
%

%
% ---------------
% EXAMPLE TABLE
%
% \begin{table}
% \caption{Time of the Transition Between Phase 1 and Phase 2$^{a}$}
% \centering
% \begin{tabular}{l c}
% \hline
%  Run  & Time (min)  \\
% \hline
%   $l1$  & 260   \\
%   $l2$  & 300   \\
%   $l3$  & 340   \\
%   $h1$  & 270   \\
%   $h2$  & 250   \\
%   $h3$  & 380   \\
%   $r1$  & 370   \\
%   $r2$  & 390   \\
% \hline
% \multicolumn{2}{l}{$^{a}$Footnote text here.}
% \end{tabular}
% \end{table}

%%%%%%%%%%%%%%%%%%%%%%%%%%%%%%%%%%%%%%%%%%%%%%%
% SIDEWAYS FIGURES and TABLES
% AGU prefers the use of {sidewaystable} over {landscapetable} as it causes fewer problems.
%
% \begin{sidewaysfigure}
% \includegraphics[width=20pc]{figsamp}
% \caption{caption here}
% \label{newfig}
% \end{sidewaysfigure}
%
%  \begin{sidewaystable}
%  \caption{Caption here}
% \label{tab:signif_gap_clos}
%  \begin{tabular}{ccc}
% one&two&three\\
% four&five&six
%  \end{tabular}
%  \end{sidewaystable}

%% If using numbered lines, please surround equations with \begin{linenomath*}...\end{linenomath*}
%\begin{linenomath*}
%\begin{equation}
%y|{f} \sim g(m, \sigma),
%\end{equation}
%\end{linenomath*}

%%% End of body of article

%%%%%%%%%%%%%%%%%%%%%%%%%%%%%%%%%%%%%%%%%%%%%%%
%% Optional Appendices go here
%
% The \appendix command resets counters and redefines section heads
%
% After typing \appendix
%
%\section{Here Is Appendix Title}
% will show
% A: Here Is Appendix Title
%
%\appendix
%\section{Here is a sample appendix}

%%%%%%%%%%%%%%%%%%%%%%%%%%%%%%%%%%%%%%%%%%%%%%%
% Optional Glossary, Notation or Acronym section goes here:
%
% Glossary is only allowed in Reviews of Geophysics
%  \begin{glossary}
%  \term{Term}
%   Term Definition here
%  \term{Term}
%   Term Definition here
%  \term{Term}
%   Term Definition here
%  \end{glossary}


%%%%%%%%%%%%%%%%%%%%%%%%%%%%%%%%%%%%%%%%%%%%%%%
% Acronyms
%% NOTE that acronyms in the final published version will be spelled out when used in figure captions.
%   \begin{acronyms}
%   \acro{Acronym}
%   Definition here
%   \acro{EMOS}
%   Ensemble model output statistics
%   \acro{ECMWF}
%   Centre for Medium-Range Weather Forecasts
%   \end{acronyms}


%%%%%%%%%%%%%%%%%%%%%%%%%%%%%%%%%%%%%%%%%%%%%%%
% Notation
%   \begin{notation}
%   \notation{$a+b$} Notation Definition here
%   \notation{$e=mc^2$}
%   Equation in German-born physicist Albert Einstein's theory of special
%  relativity that showed that the increased relativistic mass ($m$) of a
%  body comes from the energy of motion of the body—that is, its kinetic
%  energy ($E$)—divided by the speed of light squared ($c^2$).
%   \end{notation}




%%%%%%%%%%%%%%%%%%%%%%%%%%%%%%%%%%%%%%%%%%%%%%%
%
% DATA SECTION and ACKNOWLEDGMENTS
%
%%%%%%%%%%%%%%%%%%%%%%%%%%%%%%%%%%%%%%%%%%%%%%%

\section*{Open Research Section}
This section MUST contain a statement that describes where the data supporting the conclusions can be obtained. Data cannot be listed as ''Available from authors'' or stored solely in supporting information. Citations to archived data should be included in your reference list. Wiley will publish it as a separate section on the paper’s page. Examples and complete information are here:
https://www.agu.org/Publish with AGU/Publish/Author Resources/Data for Authors

\section*{As Applicable – Inclusion in Global Research Statement}
The Authorship: Inclusion in Global Research policy aims to promote greater equity and transparency in research collaborations. AGU Publications encourage research collaborations between regions, countries, and communities and expect authors to include their local collaborators as co-authors when they meet the AGU Publications authorship criteria (described here: https://www.agu.org/publications/authors/policies\#authorship). Those who do not meet the criteria should be included in the Acknowledgement section. We encourage researchers to consider recommendations from The TRUST CODE - A Global Code of Conduct for Equitable Research Partnerships (https://www.globalcodeofconduct.org/) when conducting and reporting their research, as applicable, and encourage authors to include a disclosure statement pertaining to the ethical and scientific considerations of their research collaborations in an “Inclusion in Global Research Statement’ as a standalone section in the manuscript following the Conclusions section. This can include disclosure of permits, authorizations, permissions and/or any formal agreements with local communities or other authorities, additional acknowledgements of local help received, and/or description of end-users of the research. You can learn more about the policy in this editorial. Example statements can be found in the following published papers: 
Holt et al. (https://agupubs.onlinelibrary.wiley.com/doi/full/10.1029/2022JG007188), 
Sánchez-Gutiérrez et al. (https://agupubs.onlinelibrary.wiley.com/doi/abs/10.1029/2023JG007554), 
Tully et al. (https://agupubs.onlinelibrary.wiley.com/doi/epdf/10.1029/2022JG007128) 
Please note that these statements are titled as “Global Research Collaboration Statements” from a previous pilot requirement in JGR Biogeosciences. The pilot has ended and statements should now be titled “Inclusion in Global Research Statement”.



\acknowledgments
Enter acknowledgments here. This section is to acknowledge funding, thank colleagues, enter any secondary affiliations, and so on.


%%%%%%%%%%%%%%%%%%%%%%%%%%%%%%%%%%%%%%%%%%%%%%%
% REFERENCES and BIBLIOGRAPHY
%
\bibliography{refs} 
%%%%%%%%%%%%%%%%%%%%%%%%%%%%%%%%%%%%%%%%%%%%%%%





%Reference citation instructions and examples:
%
% Please use ONLY \cite and \citeA for reference citations.
% \cite for parenthetical references
% ...as shown in recent studies (Simpson et al., 2019)
% \citeA for in-text citations
% ...Simpson et al. (2019) have shown...
%
%
%...as shown by \citeA{jskilby}.
%...as shown by \citeA{lewin76}, \citeA{carson86}, \citeA{bartoldy02}, and \citeA{rinaldi03}.
%...has been shown \cite{jskilbye}.
%...has been shown \cite{lewin76,carson86,bartoldy02,rinaldi03}.
%... \cite <i.e.>[]{lewin76,carson86,bartoldy02,rinaldi03}.
%...has been shown by \cite <e.g.,>[and others]{lewin76}.
%
% apacite uses < > for prenotes and [ ] for postnotes
% DO NOT use other cite commands (e.g., \citet, \citep, \citeyear, \nocite, \citealp, etc.).
%



\end{document}



More Information and Advice:

%%%%%%%%%%%%%%%%%%%%%%%%%%%%%%%%%%%%%%%%%%%%%%%
%
%  SECTION HEADS
%
%%%%%%%%%%%%%%%%%%%%%%%%%%%%%%%%%%%%%%%%%%%%%%%

% Capitalize the first letter of each word (except for
% prepositions, conjunctions, and articles that are
% three or fewer letters).

% AGU follows standard outline style; therefore, there cannot be a section 1 without
% a section 2, or a section 2.3.1 without a section 2.3.2.
% Please make sure your section numbers are balanced.
% ---------------
% Level 1 head
%
% Use the \section{} command to identify level 1 heads;
% type the appropriate head wording between the curly
% brackets, as shown below.
%
%An example:
%\section{Level 1 Head: Introduction}
%
% ---------------
% Level 2 head
%
% Use the \subsection{} command to identify level 2 heads.
%An example:
%\subsection{Level 2 Head}
%
% ---------------
% Level 3 head
%
% Use the \subsubsection{} command to identify level 3 heads
%An example:
%\subsubsection{Level 3 Head}
%
%---------------
% Level 4 head
%
% Use the \subsubsubsection{} command to identify level 3 heads
% An example:
%\subsubsubsection{Level 4 Head} An example.
%
%%%%%%%%%%%%%%%%%%%%%%%%%%%%%%%%%%%%%%%%%%%%%%%
%
%  IN-TEXT LISTS
%
%%%%%%%%%%%%%%%%%%%%%%%%%%%%%%%%%%%%%%%%%%%%%%%
%
% Do not use bulleted lists; enumerated lists are okay.
% \begin{enumerate}
% \item
% \item
% \item
% \end{enumerate}
%
%%%%%%%%%%%%%%%%%%%%%%%%%%%%%%%%%%%%%%%%%%%%%%%
%
%  EQUATIONS
%
%%%%%%%%%%%%%%%%%%%%%%%%%%%%%%%%%%%%%%%%%%%%%%%

% Single-line equations are centered.
% Equation arrays will appear left-aligned.

Math coded inside display math mode \[ ...\]
 will not be numbered, e.g.,:
 \[ x^2=y^2 + z^2\]

 Math coded inside \begin{equation} and \end{equation} will
 be automatically numbered, e.g.,:
 \begin{equation}
 x^2=y^2 + z^2
 \end{equation}


% To create multiline equations, use the
% \begin{eqnarray} and \end{eqnarray} environment
% as demonstrated below.
\begin{eqnarray}
  x_{1} & = & (x - x_{0}) \cos \Theta \nonumber \\
        && + (y - y_{0}) \sin \Theta  \nonumber \\
  y_{1} & = & -(x - x_{0}) \sin \Theta \nonumber \\
        && + (y - y_{0}) \cos \Theta.
\end{eqnarray}

%If you don't want an equation number, use the star form:
%\begin{eqnarray*}...\end{eqnarray*}

% Break each line at a sign of operation
% (+, -, etc.) if possible, with the sign of operation
% on the new line.

% Indent second and subsequent lines to align with
% the first character following the equal sign on the
% first line.

% Use an \hspace{} command to insert horizontal space
% into your equation if necessary. Place an appropriate
% unit of measure between the curly braces, e.g.
% \hspace{1in}; you may have to experiment to achieve
% the correct amount of space.


%%%%%%%%%%%%%%%%%%%%%%%%%%%%%%%%%%%%%%%%%%%%%%%
%
%  EQUATION NUMBERING: COUNTER
%
%%%%%%%%%%%%%%%%%%%%%%%%%%%%%%%%%%%%%%%%%%%%%%%

% You may change equation numbering by resetting
% the equation counter or by explicitly numbering
% an equation.

% To explicitly number an equation, type \eqnum{}
% (with the desired number between the brackets)
% after the \begin{equation} or \begin{eqnarray}
% command.  The \eqnum{} command will affect only
% the equation it appears with; LaTeX will number
% any equations appearing later in the manuscript
% according to the equation counter.
%

% If you have a multiline equation that needs only
% one equation number, use a \nonumber command in
% front of the double backslashes (\\) as shown in
% the multiline equation above.

% If you are using line numbers, remember to surround
% equations with \begin{linenomath*}...\end{linenomath*}

%  To add line numbers to lines in equations:
%  \begin{linenomath*}
%  \begin{equation}
%  \end{equation}
%  \end{linenomath*}



