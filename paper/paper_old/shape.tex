\documentclass[12pt,a4paper]{article}

\usepackage[dvipsnames]{xcolor}
\usepackage{amsmath}
\usepackage{amsfonts}
\usepackage[T1]{fontenc} %can also add OT1
\usepackage{tikz}
\tikzset{every picture/.style={line width=0.75pt}} %set default line width to 0.75pt 
\usepackage{siunitx}
\usepackage[style=apa,backend=biber]{biblatex}
\addbibresource{refshape.bib}

\graphicspath{{figures/}}
   
\usepackage{lineno}
\linenumbers
   
\usepackage{float} %for figure placement

\usepackage{draftwatermark}
\SetWatermarkLightness{0.8}
\SetWatermarkAngle{-40}
\SetWatermarkScale{2}
\SetWatermarkFontSize{2cm}
\SetWatermarkText{\textbf{Preprint}}
       
%%COMMANDS
\newcommand{\pard}[2]{\frac{\partial #1}{\partial #2}}
\newcommand{\vct}[1]{\underline{#1}}
\newcommand{\tsr}[1]{\underline{\underline{#1}}}


%TODO DEMAIN MATIN TOT (avant 9h isl):
%
%CONCLUSION OK
%RESULT
%CORRECT ERRORS ENGLISH AND SPELLING
%REREAD
%
%TODO DEMAIN MATIN 9-11h
%
%REREAD REPORT
%INCLUDE REFS
%INCLUDE ARTICLE
%
%ASK FRESYTEINN :
%USEFULL TO DO PROCEDURE DESCRIBED METHOD ???
%WHAT NEXT ?
%
%
%TODO APRES
%
%ADDITIONAL TESTS
%MODIF ARTICLE
%SUPP INFO





\begin{document}
	
\begin{center}	
	
	\huge{Shape optimization for improved understanding of magmatic plumbing systems} \\ 
	 \vspace{1cm}
	\normalsize{ 
	\textbf{Théo Perrot, Freysteinn Sigmundsson} \\
	June 2024}
\end{center}


\section*{Abstract}
In volcano geodesy, inverse problems caused by identifying the location and shape of magmatic bodies based on ground deformation data are common. Traditional approaches often rely on models with predefined shapes, which can limit their accuracy. To address this, we present a shape optimisation method using a level-set approach that flexibly determines the optimal shape of a magma chamber without prior shape assumptions. By minimising the discrepancy between observed and modelled surface displacements, our adapted algorithm becomes suitable for solving inverse volcano deformation problems. We explore the capabilities of this approach with synthetic data and apply it to InSAR observations of the Svartsengi volcanic system in Iceland, demonstrating its potential to improve volcanic hazard assessment after maturation through future work.







\newpage



\section{Introduction}

\subsection{Challenge}

In volcano geodesy, inverse problems are central to estimating the position of magmatic bodies using ground motion as a proxy. Displacement is observed by geodetic measurements such as Global Navigation Satellite System (GNSS) point positioning, leveling campaigns, or Synthetic Aperture Radar (InSAR) interferometry within a volcanic field, and the subsurface processes causing the movement are inferred from these observations (\cite{dzurisin2007a}). Magmatic sources are modeled as pressurized cavities that deform the surrounding host rocks and cause the surface to move. Various inversion methods based on parametric analytical or numerical models aim at finding the optimal values for the vector of $d$ free parameters $\vec{m} \in \mathbb{R}^d$ of the model. Then an error function $J(\vec{m})$ is representative of the misfit between the observed displacements and the prediction of the model. $\vec{m}_{opt}$ can then be found using various inversion techniques that minimize $J$: global optimization based on analytic (\cite{cervelli2001}) or numerical models (\cite{hickey2014a}, \cite{charco2014}), Bayesian inference (\cite{bagnardi2018a}, \cite{trasatti2022}), or genetic algorithms (\cite{velez2011}) on analytic models. The choice of the method is constrained by the reasonable number of evaluations of $J(\vec{m})$: numerical models handle a complex description of the system, but are computationally expensive compared to analytic models, which on the other hand may lead to an oversimplification (\cite{taylor2021}).

However, each of these finite-dimensional optimization methods is limited by the intrinsic assumption of a definite parametric shape for the source. In fact, analytic expressions can be derived for only a few regular shapes such as point source (\cite{mogi1958}), finite sphere source (\cite{mctigue1987a}), or ellipsoidal source (\cite{yang1988}), and any numerically generated shape must be parameterized to be inverted. Even in the case where complex shapes are chosen, they would require additional describing parameters, and ultimately any of the above methods may face the curse of dimensionality. The goal of this paper is not to give a definitive answer to these limitations, but rather to lay the first stone for a new approach that overcomes these difficulties.

\subsection{Shape optimization}
\label{sec:th}

Shape (and topology) optimization aims to find the shape that minimizes a given function defined on a given system, without the need for prior assumptions about shape and topology. It is actively developed by part of the applied mathematics community and is widely used in engineering to find optimal designs for systems: In structural mechanics, to maximize the stiffness of a solid structure such as a cantilever beam (\cite{bendsoe2004}), in fluid-structure interaction on heat exchangers or flying obstacles (\cite{feppon2020}), and even as a way to explore new architecture for buildings (\cite{beghini2014}).Most finite element simulation and design software now implements an embedded shape optimization module (\cite{frei2015}, \cite{slavov2019}, \cite{lequilliec2014}). However, its use has not yet been reported in the context of inverse problems in volcano geodesy, where it can overcome the shape hypothesis problem as long as an internal pressure value is assumed.


Many paradigms coexist in shape optimization as reviewed by \cite{sigmund2013a}, one of the most popular being SIMP optimization, where a density value is optimized for each element of the mesh with values between 0 (void) and 1 (material) before being black and white filtered to output a design (\cite{sigmund2001},\cite{bendsoe2004}), with several open source implementations (\cite{andreassen2011a},\cite{hunter2017}). We chose level-set shape optimization instead because it has the advantage of providing an explicit representation of the boundary at each step of the optimization, which is crucial for us as explained later (section \ref{sec:mag}). For this, we relied on the work of \cite{dapogny2023}, who thoroughly described and vulgarized the method, as well as providing a freely available open source implementation of the method, \texttt{sotuto}(\cite{dapogny2024}), which we modified and extended to adapt it to inverse geodetic problems.




\section{Method}
\label{sec:mag}

Here we briefly present the key ingredients of level set shape optimization along with their implications for our problem. The full mathematical background on which it relies is not detailed, but see this chapter by \cite{allaire2021} for a comprehensive step-by-step description supported by proofs and theorems. It is also worth noting that many aspects of secondary importance to the method are not mentioned for the sake of brevity. For the unfamiliar reader interested in understanding the method, the lecture (especially part III) given by \cite{dapogny} at the Université Grenoble Alpes is also recommended.


\subsection{Model}

\begin{figure}[h!]
	\centering
	\includegraphics[width=\textwidth]{magdiag.png}
	\caption{2D sketch of the problem. The optimized boundary (where the level-set function is zero) is the magma chamber wall $\Gamma_{s}$ subjected to a uniform normal load $\sigma(u).n = f_s $ on $\Gamma_s$, where $f_s = - \Delta P . n$, where $n$ is the unit normal vector and $\Delta P$ is the pressure change between the magma source and the surrounding crust. The bottom surface $\Gamma_{b}$ is fixed ($u=0$). The other boundaries are free. The target displacement field $u_o$ is known on the upper surface $\Gamma_{u}$. }
	\label{fig:mod}
\end{figure}



Let $\Omega$ be a bounded domain of $\mathbb{R}^3$ whose shape we want to optimize by modifying parts of its boundary $\partial \Omega$. As for classical analytical models of volcanic deformation induced by magmatic activity, $\Omega$ is a domain representing a portion of the shallow Earth crust, including the volcano, assumed to be homogeneous, isotropic, and elastic. The governing equations are $-div(Ae(u))=0 \text{ in } \Omega$, where $e(u)$ is the strain tensor of the displacement field $u$ and $A$ is the constitutive law tensor, $Ae = 2\mu e + \lambda tr(e)Id$ for linear elasticity. Boundaries under different conditions, see Fig. \ref{fig:mod} for all notations.

The part of $\partial \Omega$ to be optimized is $\Gamma_{s}$, the boundary magma chamber, which is modeled as an empty, uniformly pressurized cavity. Therefore, a value for the internal pressure $\Delta P$ must be assumed (see Discussion for development). In the following text we talk about optimizing $\partial \Omega$, but in practice only $\Gamma_{s} \subset \partial \Omega$ is of interest and will be modified, any other boundary will be fixed during the iterations.



We want to find $\partial \Omega$ such that the displacement of the model $u(\Omega)$ is as close as possible to the observed displacement $u_o$ on the surface $\Gamma_{u}$. Thus, the unconstrained shape optimization problem we want to solve is the minimization of a squared RMS discrepancy

\begin{equation}\label{eq:min}
	\min_{\Omega} J(\Omega) = \int_{\Gamma_u}(u(\Omega)-u_o)^2 dS
\end{equation}





\subsection{Hadamard Boundary Variation}

\begin{figure}[h!]
	\centering
	\includegraphics[width=0.5\textwidth]{boundvar.png}
	\caption{Reproduced from \cite{allaire2021}}
	\label{fig:var}
	
\end{figure}

Overall, this method can be considered a classical iterative gradient descent algorithm. $J$ is first initialized at $J_0$ with an instructed first guess for $\Omega_0$ and then iteratively decreased by moving $\partial \Omega$ of a given step in a given descent direction $\theta: \mathbb{R}^3 \mapsto\mathbb{R}^3 \in W^{1,\inf}$ (the Sobolev space of uniformly bounded functions, \cite{allaire2021}) chosen using the shape derivative $J'(\Omega)(\theta)$.

The boundary variation method of \cite{hadamard1908} introduces the notion of shape differentiation $F'(\Omega)(\theta)$ of a functional $F$ defined on $\Omega$ in the direction $\theta$. In short, such a derivative is based on the variation of a bounded domain $\Omega \mapsto \Omega_{\theta} := (Id + \theta)(\Omega)$: the surface $\partial \Omega$ is slightly moved according to a small vector field $\theta(x)$, as shown in Fig. \ref{fig:var}. Once such a derivative exists, one can compute a descending direction at the n$^{th}$step $\theta_n$, such as $J'(\Omega)(\theta_n)\leq0$, so $J_{n+1}\leq J_n$, to decrease the value of $J$ at each iteration.



In our case, after derivation based on the \cite{cea1986} formal method, we found under the variational form :
\begin{equation}\label{eq:jp}
	J'(\Omega)(\theta) = \int_{\Gamma_s}\left(Ae(u):e(p) +\pard{f_s}{n}p+\pard{p}{n}f_s+\kappa f_s p \right ).\theta .n dS
\end{equation}

where $\kappa=div(n)$ is the mean curvature at the boundary, and $p$ is the adjoint solution of

\begin{eqnarray}
	\forall v \in H^1(\mathbb{R}^3), \int_{\Gamma_u}2(u_\Omega-u_o)vdS + \int_{\Omega}Ae(v):e(p)dV =0
	\\ \mathrm{and}~ p=0 \text{ on } \Gamma_{b} \nonumber
\end{eqnarray}

From there, we can trivially move $\Omega$ in the direction $\theta = -A$ (where $A$ is the integrand term in parentheses) to ensure that $J'(\Omega)(\theta) \leq 0$. This guarantees that $J(\Omega_{n+1}) \leq J(\Omega_{n})$: the series ${J(\Omega_{n})}$ converges to a minimum.



\subsection{Level-set representation}

A key issue is the representation of the surface to be optimized. The level set method allows to track dramatic changes as well as topology variations (creation of new holes). A certain function $\phi:D\mapsto\mathbb{R}$ is defined over the domain $D \in \mathbb{R}^3 $ in such a way that the shape boundary is the level set $0$, i.e. reads $\partial \Omega = \phi(x=0)$. Basically, $\phi$ can be taken as the signed distance between any point $x$ and $\partial \Omega$, as shown in the example fig. \ref{fig:phi}. In this way, $\partial \Omega$ is implicitly manipulated when transforming $\phi$.

\begin{figure}[h!]
	\centering
	\includegraphics[width=0.8\textwidth]{phi.png}
	\caption{Reproduced from \cite{dapogny2023}}
	\label{fig:phi}

\end{figure}

$\Omega_n$ is then deformed by advecting the corresponding $\phi_n$ with a velocity field $V(x) = \tau_n \theta_n$, where $\tau_n$ is the additional step size. The advection equation usually appears in fluid mechanics to describe the evolution of a quantity transported by a given velocity field, but here there is a smooth and flexible way to modify $\phi$ which ensures smoothness of $\Omega_{n+1}$ and change of topology (see \cite{allaire2021}).



\subsection{Numerical implementation}


In practice, the $D$ domain is discretized into a mesh $T_n$ on which each variational form is solved at each iteration $n$. This includes the solution of the elasticity to get $u_n$, the adjoint state $p_n$, the computation of the shape gradient $J'_n$, the descent direction $\theta_n$, the advection of $\phi_n$. In \texttt{sotuto} it is achieved by calling scripts written in FreeFem++, a finite element software that allows solving any integral form of elliptic PDE (\cite{hecht2012}).


Once the new form $\Omega_{n+1}$ is computed and discretized thanks to a local remeshing phase, a new evaluation of $J^{n+1}$ is performed. Since $\tau$ is arbitrarily fixed and initialized to $1$, it can happen that $\Omega_{n}$ is shifted by too large a step and so $J_{n+1} is \geq J_{n}$. To adjust the step size, a line search procedure is implemented and adjusts the step size by decreasing it if the new iteration is the worst to ensure an improvement of $J$ by computing a new $\Omega_{n+1}$ being a less deformed version of $\Omega_{n}$. On the contrary, if $\Omega_{n+1}$ is accepted, $\tau$ is increased to speed up convergence. A tolerance is set to accept iterations if the increase in $J$ is reasonable.

The global optimization loop has no termination criterion. Thus, it is up to the user to stop it when no significant improvement in $J$ can be achieved, or when the shape is not realistic.

The loop and the line search are implemented in Python in \texttt{sotuto}. Then the FreeFem scripts are called by the Python script core and data is exchanged via temporary files.



The above aspects are implemented in \texttt{sotuto}. However, we extended its functionality to handle our geophysical problem, in a fork we called \texttt{magmaOpt}. This included: scripts to create the domain and initial source with a flexible mesher GMSH which handle complex geometries such as the one generated by topography \cite{geuzaine2009a} ,adapting FreeFem scripts to different error functions, allowing optimization of the loaded boundary $\Gamma_{s}$ and so on.




\section{Validation with synthetic data}
\label{sec:synth}

To test the method, the idea is to do a kind of cross-validation. On the one hand, we form synthetic observation data from a known source.  On the other hand, we initialized the algorithm with a first guess for the source shape and location. We expect the algorithm to iteratively modify the shape of the source and converge to the correct shape and location.  In fact, the 3D location of the source (e.g., its center of gravity for a random shape) is not directly optimized as a vector of discrete parameters, but is modified by the simple fact that the boundary is free to move in any direction, and thus can take on a kind of "average rigid body motion" as it gradually moves the center in a given direction.

In practice, the synthetic observed surface displacement field is derived from the \cite{mctigue1987a} solution, an analytical approximation of the displacement caused by a uniformly pressurized spherical cavity (the magma domain) embedded in an isotropic, homogeneous, and planar elastic medium (the host crust) with elastic constants tp $E=10\si{GPa}$ and $\mu=0.25$.

Usually, the quantities to be determined with parametric inversion based on a McTigue model are the location and the radius. The pressure change can also be left as a free parameter, but is interchangeable with the radius, so one must be fixed to determine the other, see (\cite{greiner2021a}) for more details. For the synthetic source, we fixed these free parameters to $z=-5\si{km}$, $\Delta P = 10\si{MPa}$, $R=1.5\si{km}$, which are typical values for inverted magmatic domains.




\texttt{magmaOpt} is then allowed to run freely, without any termination condition, to see whether or not it succeeds in converging from the ellipsoid to the McTigue sphere we used to generate the synthetic displacement.

% TODO : talk with charles about that
%To test the robustness of the method, we tried different initial $\Omega_0$ shapes. The tests included: same center but different shape, different center but same shape, and different shape and center, as summarized in the table \ref{tab:res}.
%
%Since there is no termination condition here, we take the shape corresponding to the smallest value of $J$ found as the best result (although it can be discussed, see the conclusion). To evaluate the quality of the result, we use two different metrics. On the one hand, the distance between the centroids of the final source and the synthetic sphere is an indicator of the distance between the two corps and assesses whether the final shape taken as a rigid body is overall correctly located. On the other hand, after aligning the final source surface on the sphere using an iterative point cloud (ICP) algorithm, we use the chamfer distance to asses the difference in terms of shape.

% 
%TO DISCUSS
% \begin{table}[!htb]
	% 	\centering
	% 	\begin{tabular}{|c|c|c|c|}
		% 		\hline
		% 		Initial location &    Initial shape     &   Final centroid distance & Final chamfer distance  \\
		% 		\hline  
		% 		\hline 
		% 		AAAA&    AAAA &          AAAA     &  AAAA \\ \hline
		% 
		% 	\end{tabular}
	% 	
	% 	\caption{Centroid and shape distances between different initialization $\Omega_0$ and the objective sphere of parameters $x=y=0,z=-5\si{km},R=1.5\si{km}$ for $\Delta P=10\si{MPa}$.}
	% 	\label{tab:res}
	% \end{table}
	
	
\begin{figure}[h!]	
	\centering
	\includegraphics[width=1.1\textwidth]{synthopt.png}
	\caption{Evolution of error and succesivve shapes taken by the magma source during an optimization loop. The initial guess is a flat ellipsoid of semi-axes $r_x=2\si{km},r_y=3\si{km},r_z=1\si{km}$} centered on the true spherical source. The minimum is reached at iteration 82.
	\label{fig:synth}
	
\end{figure}
 
 
 
 As shown in the figure \ref{fig:synth}, the algorithm seems to converge to a minimum. After that, the slope of the cost function is positive because a small increase in $J$ is allowed. It is obvious that no other minima are found, as the shape evolves towards a stick-shaped feature, far from the expected solution. We can also discuss the minima found. The surface reached is obviously not a sphere, but it is closer than any shape found before. We expect the shape to be closer to a sphere with a finer mesh defined. Many improvements could be realized: for example, once it is obvious that the algorithm will not converge to a better solution, we could restart the algorithm on the best solution found, set new evolution parameters, and allow a finer mesh. By repeating this process automatically, it may be possible to arrive at a more likely shape for the magma reservoir.
 

 





\section{Real test case : Svartsengi 2022 inflation}
\label{sec:real}

We now apply the method to infer the shape of a magma domain in a recent period of volcanic unrest and eruption in SW Iceland by evaluating the shape of a magma body responsible for the ground inflation observed from 21 April to 14 June 2022 at Svartsengi on the Reykjanes peninsula.  This is one of 5 inflation episodes that preceded catastrophic dike breaches and eruptions at the Sundhnúkur crater row, which caused the destruction of the city of Grindavík (\cite{sigmundsson2024b}).

The observational data used are the line-of-sight (LOS) displacement maps of the area from Cosmo SkyMed available in \cite{parks2024}, the data used in \cite{sigmundsson2024b}. After uniform downsampling and mesh reprojection (the data points must be aligned on the mesh nodes), the ascending A32 and descending D132 tracks were both used in the RMS error function we adapted to the LOS geometry.
\begin{equation}
	J(\Omega)=\sum_{i\in {tck}} \alpha_i \int_{\Gamma_{u}}\left(L_{i}(u(x))-l^{i}_o(x)\right)^2dS
\end{equation}
Where $tck=\{A125,D132\}$. For each track $i$, $\alpha_i$ is the weight of the track ($\forall i, \alpha_i= 1$ here), $L_i:\mathbb{R}^3 \mapsto \mathbb{R}$ is the function that projects the 3D surface displacement given by the model into the LOS geometry, and $l^{i}_o$ is the observed LOS displacement.

We then used the framework developed above, only projecting the InSAR data onto the mesh of $D$ and modifying the expression of the error function in \texttt{magmaOpt}.

\begin{figure}[H]
	\centering
	\includegraphics[width=0.8\textwidth]{resinsar.png}
	\caption{a) Convergence plot with embedded zoom. The blue line is the error and the orange line is the evolution of $\tau$. Minima are reached at iteration 128. b,c) Side and top view of the source $\Gamma_{s}$ minimizing $J$. d) Data, model and residuals of the LOS displacements at iteration 128 for the two InSAR tracks A32 (top) and D132 (bottom). Black arrows are heading and looking directions, coordinates are ISN16 \cite{islands} shifted to a local origin $(2529373 E,179745 N)$. }
	\label{fig:rsar}

\end{figure}

The results shown in figure \ref{fig:rsar} are encouraging: after providing an initial guess located at the center of inflation at depth for a sphere of radius RR, the algorithm is able to iteratively change the shape and depth of the magma domain to finally result in a sill-like flattened spheroid whose centroid is located at DD depth. This is consistent with the presumed depth found in the supporting information of \cite{sigmundsson2024b}, which performs an analytical model-based inversion. Although the pressure must be fixed, as explained in \ref{sec:th}, the result can be used to compare the final shape of the magmatic intrusion and give a richer insight into it. Here we see interesting features, such as an increasing thickness on the north side, that can't be traced by any other method. The algorithm produces features that we consider to be artifacts, probably due to mesh refinement problems, such as small holes or horn-shaped features.


\section{Discussion}

This work paves the way for a new class of methods that tackle an unknown geometry of the magmatic domains, thus giving the possibility to explore irregular shapes that are more likely to exist compared to any other usually assumed regular shapes.
However, even if the first results presented are promising, many questions remain to be answered.
First of all, the internal pressure of the chamber must be specified, which is a strong hypothesis. In this context, the precise shape of the source should be determined as a second step. The traditional analytical model-based inversion would be run first, giving a pressure and a first educated guess for the position and shape of $\Omega_0$. Then a more realistic shape could be sought with a shape optimization taking the output of the inversion as an initial guess.

Adding constraints may also be an interesting way to explore. For example, the volume of the source could be constrained to be within bounds or even to match a certain value. The implemented shape optimization is certainly able to handle constraints as described by \cite{allaire202}. The physical meaning of the best shape might benefit from a more constrained problem, and the less influencing deeper part of the source might be less random.

To better understand the influence of data partitioning and variablitiy, additional tests could be run with synthetic data. We can think of tests such as masking part of the surface displacement field, introducing noise and parasitic signals, reducing the number of data points, as is often the case in reality with areas of volcanic systems lacking data coverage (glacier, river, lava, forest) and subjected to perturbations (atmospheric distortion, weather).

It is also important to mention that the behavior of the algorithm is influenced by numerous parameters of varying importance, starting from the discretization length (element size) or the domain extent, to the limits of the step size $\tau$, the regularization length, or the number of iterations allowed by the line search. A systematic study of each of these parameters would be beneficial in assessing the quality of the shape inferred.

Exploring a way to quantify the uncertainty of the answer is also crucial. For example, a sensitivity analysis approach could be considered, as well as the inclusion of a probabilistic quantity.


\section{Conclusion}


The present study has successfully demonstrated the application of inverse problems and computational methods to infer the shape of a magma domain beneath a volcano using ground inflation data from satellite observations. The shape optimization technique used in this research showed a new way to identify the most likely shape of the magma chamber. It was intended as an opening to new methods rather than a complete solution.

We have shown that modifying a shape optimization algorithm to handle geophysical problems is feasible and of interest. Tests on synthetic data showed to some extent the relevance of the approach, although the best sources found exposed the limitations faced by these first attempts of shape optimization for volcano geodesy. The test on real data showed a concrete case of how the method could be used after being more mature.

The perspectives are numerous. The numerical nature of the models allows to easily add complexities to the modeling, such as complex mechanical behavior of the crust (plaxticity, viscoelasticity, poroelsaticity), additional loads (tectonic stress, tidal loads, glacier weights), or inhomogeneities. We hope that the open source code \texttt{magmaOpt} developed by us will be modified and extended by future work.

By addressing these limitations and extending this approach, researchers can further improve the accuracy and reliability of magma domain shape inference. Ultimately, the development of more sophisticated models will enable geophysicists to better monitor volcanic activity, predict eruptions, and provide critical support for hazard mitigation strategies.







\printbibliography



\end{document}
